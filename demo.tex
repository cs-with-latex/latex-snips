\documentclass[UTF8,a4paper,11pt]{ctexart}
\usepackage{listings} 
\usepackage{xcolor} 
\usepackage{amsmath}
\lstset{
  basicstyle=\tt,
  keywordstyle=\color{purple}\bfseries,
  identifierstyle=\color{brown!80!black},
  commentstyle=\color{gray},
  showstringspaces=false,
}
\title{自动展开功能演示}
\author{5eqn}
\date{\today}
\begin{document}
  \maketitle
  \section{抄来的功能}
    自动下标:
    \begin{lstlisting}[language=] 
      a3+a2=a5
      pi3=Pi3\end{lstlisting}
    \[
    \begin{aligned}
      a_3+a_2=a_5
      \\\pi_3=\Pi_3
    .\end{aligned}
    \]
    
    自动分数:
    \begin{lstlisting}[language=] 
      1/2
      pi/1+2<TAB>+3
      (1+((1+4)+5))/1+4
      //114<TAB>514\end{lstlisting}
    \[
    \begin{aligned}
      \frac{1}{2}
      \\\frac{\pi}{1+2}+3
      \\\frac{1+((1+4)+5)}{1+4}
      \\\frac{114}{514}
    .\end{aligned}
    \]
    
  \section{预制函数自动括号}
    分数中有减号不形成括号: 
    \begin{lstlisting}[language=]
      sinpi/2-x
      asin//lambda+1<TAB>mu+1\end{lstlisting}
    \[
    \begin{aligned}
      \sin \frac{\pi}{2-x}
      \\ \arcsin \frac{\lambda+1}{\mu+1}
    .\end{aligned}
    \]

    需要括号的时候自动括号: 
    \begin{lstlisting}[language=] 
      sinpi/2<TAB>-x
      atanxi1+xi2\end{lstlisting}
    \[
    \begin{aligned}
      \sin\left(\frac{\pi}{2}-x\right)
      \\ \arctan \left(\xi_1+\xi_2\right)
    .\end{aligned}
    \]

    不需要括号的时候用空格表示: 
    \begin{lstlisting}[language=] 
      sinpi/2<TAB> -x
      atan2 -x\end{lstlisting}
    \[
    \begin{aligned}
      \sin \frac{\pi}{2} -x
      \\ \arctan 2 -x
    .\end{aligned}
    \]
    
    对 sin 追加平方: 
    \begin{lstlisting}[language=] 
      sinpi/2<TAB>-x<TAB>psr
      sinx psr\end{lstlisting}
    \[
    \begin{aligned}
      \sin^2\left(\frac{\pi}{2}-x\right)
      \\ \sin^2 x 
    .\end{aligned}
    \]
    
    普通的 sin 平方: 
    \begin{lstlisting}[language=] 
      sinsrpi/2<TAB>-x
      sinsrx\end{lstlisting}
    \[
    \begin{aligned}
      \sin^{2}\left(\frac{\pi}{2}-x\right)
      \\ \sin ^{2}x
    .\end{aligned}
    \]
    
    对 sin 的参数平方: 
    \begin{lstlisting}[language=] 
      sinpi/2<TAB>-x<TAB>sr
      sinxsr\end{lstlisting}
    \[
    \begin{aligned}
      \sin\left(\frac{\pi}{2}-x\right)^{2}
      \\ \sin x^{2}
    .\end{aligned}
    \]
    
  \section{自定义函数和导数}
    导数: 
    \begin{lstlisting}[language=] 
      f'x+2
      Phi'3\end{lstlisting}
    \[
    \begin{aligned}
      f^{'}\left(x+2\right)
      \\ \Phi^{'}\left(3\right)
    .\end{aligned}
    \]
    
    二阶导: 
    \begin{lstlisting}[language=] 
      f''x+2
      g''3\end{lstlisting}
    \[
    \begin{aligned}
      f^{''}\left(x+2\right)
      \\ g^{''}\left(3\right)
    .\end{aligned}
    \]
    
    任意阶导: 
    \begin{lstlisting}[language=] 
      f'''n+1<TAB>x+2
      H'''n<TAB>x<TAB>+2\end{lstlisting}
    \[
    \begin{aligned}
      f^{^{(n+1)}}\left(x+2\right)\\
      H^{^{(n)}}\left(x\right)+2
    .\end{aligned}
    \]
    
    任意阶导追加平方: 
    \begin{lstlisting}[language=] 
      f'''n+1<TAB>x+2<TAB>psr
      G'''7<TAB>x<TAB>psr\end{lstlisting}
    \[
    \begin{aligned}
      f^{^{(n+1)}2}\left(x+2\right)\\
      g^{^{(7)}2}\left(x\right)
    .\end{aligned}
    \]
    
    任意阶导追加任意次方: 
    \begin{lstlisting}[language=] 
      f'''n1<TAB>x+2<TAB>ptd5
      F'''11<TAB>14<TAB>ptd45\end{lstlisting}
    \[
    \begin{aligned}
      f^{^{(n_1)}5}\left(x+2\right)
      \\ F^{^{(11)}45}\left(14\right)
    .\end{aligned}
    \]
    
  \section{线性代数}
    快速书写矩阵: 
    \begin{lstlisting}[language=] 
      pma 1 2 3  4 5 6  7 8 9 t
      Vma 1/2<TAB> 2/3<TAB>  3/4<TAB> 4/5<TAB> t\end{lstlisting}
    \[
    \begin{aligned}
      \begin{pmatrix}1&2&3\\4&5&6\\7&8&9\end{pmatrix} \\
      \begin{Vmatrix}\frac{1}{2}&\frac{2}{3}\\\frac{3}{4}&\frac{4}{5}\end{Vmatrix} 
    .\end{aligned}
    \]
  
    快速书写对角矩阵:
    \begin{lstlisting}[language=] 
      pmd 1 2 3 4 t
      bmd 1/14<TAB> 5/14<TAB> t\end{lstlisting}
    \[
    \begin{aligned}
      \begin{pmatrix}1&&&\\&2&&\\&&3&\\&&&4\end{pmatrix} \\
      \begin{bmatrix}\frac{1}{14}&\\&\frac{5}{14}\end{bmatrix} 
    .\end{aligned}
    \]
\end{document}
